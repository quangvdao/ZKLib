%!TEX root = main.tex

% ========================================================================
% Proof Environments
% ========================================================================
\newtheorem{thm}{Theorem} %[section]
\newtheorem{lem}{Lemma}
\newtheorem{cor}[thm]{Corollary}
\newtheorem{propo}[thm]{Proposition}
\newtheorem{defn}[thm]{Definition}
\newtheorem{assm}[thm]{Assumption}
\newtheorem{clm}[thm]{Claim}
\newtheorem{rem}[thm]{Remark}
\newtheorem{conj}[thm]{Conjecture}
\newtheorem{exa}{Example}

\makeatletter
\newtoks\answerscollect
%LNCS
\newcounter{question}
\setcounter{question}{0}
\def\thequestion{{\bfseries{Question \arabic{question}. }}\\}

\def\answer#1{%
\protected@edef\answertmp{%
\the\answerscollect\vspace{.5\baselineskip}\noindent\thequestion#1\par}%
\par\answerscollect=\expandafter{\answertmp}}
\def\printanswers{\the\answerscollect\answerscollect={}}
\def\initbox{\answerscollect={\par\noindent Answers:\par}}

%\newcommand{\question}[1]{\stepcounter{question}\par\noindent\thequestion#1}
\makeatother

%\iffalse
%LNCS
\newenvironment{theorem}{\begin{thm}
    \begin{sl}
    }%
    {
    \end{sl}
  \end{thm}}
%\fi
%LNCS

%\iffalse
%LNCS
\def\qsym{\vrule width0.6ex height1em depth0ex}
    \def\qedsym{{\hspace{5pt}\rule[-1pt]{3pt}{9pt}}}
    \newcount\proofqeded
    \newcount\proofended
    \def\qed{%\qedsym
    \end{rm}\addtolength{\parskip}{-0pt}
    \setlength{\parindent}{\saveparindent}
    \global\advance\proofqeded by 1 }
    \newenvironment{proof}%
     {\proofstart}%
     {\ifnum\proofqeded=\proofended\qed\fi \global\advance\proofended by 1
      \medskip}
    \makeatletter
    \def\proofstart{\@ifnextchar[{\@oprf}{\@nprf}}
    \def\@oprf[#1]{\begin{rm}\protect\vspace{6pt}\noindent{\bf Proof of #1:\
    }%
    \addtolength{\parskip}{5pt}\setlength{\parindent}{0pt}}
    \def\@nprf{\begin{rm}\protect\vspace{6pt}\noindent{\bf Proof:\ }%
    \addtolength{\parskip}{5pt}\setlength{\parindent}{0pt}}
  \makeatother
%\fi
%LNCS
%\newenvironment{corollary}{\begin{cor}\begin{sl}}%
%    {\end{sl}\end{cor}}
%\newenvironment{proposition}{\begin{propo}\begin{sl}}%
%    {\end{sl}\end{propo}}
%LNCS
\newenvironment{definition}{\begin{defn}\begin{sl}}%
    {\end{sl}\end{defn}}
%LNCS
%\newenvironment{assumption}{\begin{assm}\begin{em}}%
%    {\end{em}\end{assm}}
%\newenvironment{claim}{\begin{clm}\begin{sl}}%
 %   {\end{sl}\end{clm}}
%\newenvironment{remark}{\begin{rem}\begin{em}}%
%    {\end{em}\end{rem}}
\newenvironment{example}{\begin{exa}\begin{em}}%
    {\end{em}\end{exa}}


\newcommand{\lemautorefname}{Lemma}
\newcommand{\algorithmautorefname}{Algorithm}
%\newcommand{\subsectionautorefname}{Section}

\def\qedsym{{\hspace{5pt}\rule[-1pt]{3pt}{9pt}} }


\renewcommand{\arraystretch}{1.3}
\DeclareMathAlphabet{\mathsl}{OT1}{cmr}{m}{sl}
\DeclareMathAlphabet{\mathsc}{OT1}{cmr}{m}{sc}


\newcommand{\secref}[1]{\mbox{Section~\ref{#1}}}
\newcommand{\subsecref}[1]{\mbox{Subsection~\ref{#1}}}
\newcommand{\apref}[1]{\mbox{Appendix~\ref{#1}}}
\newcommand{\thref}[1]{\mbox{Theorem~\ref{#1}}}
\newcommand{\thmrefshort}[1]{\mbox{\textbf{Th~\ref{#1}}}}
\newcommand{\exref}[1]{\mbox{Example~\ref{#1}}}
\newcommand{\defref}[1]{\mbox{Definition~\ref{#1}}}
\newcommand{\corref}[1]{\mbox{Corollary~\ref{#1}}}
\newcommand{\lemref}[1]{\mbox{Lemma~\ref{#1}}}
\newcommand{\clref}[1]{\mbox{Claim~\ref{#1}}}
\newcommand{\propref}[1]{\mbox{Proposition~\ref{#1}}}
\newcommand{\consref}[1]{\mbox{Construction~\ref{#1}}}
\newcommand{\figref}[1]{\mbox{Figure~\ref{#1}}}
%\newcommand{\eqref}[1]{\mbox{Equation~\ref{#1}}}

\newcommand{\enumref}[1]{\mbox{(\ref{#1})}}

\newcommand{\thmlabel}[1]{\textnormal{\textbf{[#1]}}}

\newcommand{\SetFigFont}[5]{} % This is for xfig issue
\newcommand{\SetFigFontNFSS}[5]{} % This is for xfig issue

\newcommand{\headingg}[1]{{\noindent\textbf{#1.}}}
\newcommand{\heading}[1]{{\vspace{6pt}\noindent\textbf{#1.}}}
\newcommand{\proofsketch}[1]{\emph{Proof sketch}: #1}


% ========================================================================
% Lists
% ========================================================================

\newlength{\saveparindent}
\setlength{\saveparindent}{\parindent}
\newlength{\saveparskip}
\setlength{\saveparskip}{\parskip}

\newcounter{ctr}
\newcounter{savectr}
\newcounter{ectr}

\newenvironment{newitemize}{%
\begin{list}{\mbox{}\hspace{5pt}$\bullet$\hfill}{\labelwidth=15pt%
\labelsep=5pt \leftmargin=20pt \topsep=3pt%
\setlength{\listparindent}{\saveparindent}%
\setlength{\parsep}{\saveparskip}%
\setlength{\itemsep}{3pt} }}{\end{list}}


\newenvironment{newenum}{%
\begin{list}{{\rm (\arabic{ctr})}\hfill}{\usecounter{ctr} \labelwidth=17pt%
\labelsep=5pt \leftmargin=22pt \topsep=3pt%
\setlength{\listparindent}{\saveparindent}%
\setlength{\parsep}{\saveparskip}%
\setlength{\itemsep}{2pt} }}{\end{list}}

\newenvironment{tiret}{%
\begin{list}{\hspace{2pt}\rule[0.5ex]{6pt}{1pt}\hfill}{\labelwidth=15pt%
\labelsep=3pt \leftmargin=22pt \topsep=3pt%
\setlength{\listparindent}{\saveparindent}%
\setlength{\parsep}{\saveparskip}%
\setlength{\itemsep}{2pt}}}{\end{list}}

\iffalse
\newenvironment{blocklist}{%
\begin{list}{}{\labelwidth=0pt%
\labelsep=0pt \leftmargin=0pt \topsep=3pt%
\setlength{\listparindent}{\saveparindent}%
\setlength{\parsep}{\saveparskip}%
\setlength{\itemsep}{2pt}}}{\end{list}}

\newenvironment{blocklistindented}{%
\begin{list}{}{\labelwidth=0pt%
\labelsep=30pt \leftmargin=20pt\topsep=3pt%
\setlength{\listparindent}{\saveparindent}%
\setlength{\parsep}{\saveparskip}%
\setlength{\itemsep}{2pt}}}{\end{list}}
\fi
\newenvironment{onelist}{%
\begin{list}{{\rm (\arabic{ctr})}\hfill}{\usecounter{ctr} \labelwidth=18pt%
\labelsep=7pt \leftmargin=25pt \topsep=2pt%
\setlength{\listparindent}{\saveparindent}%
\setlength{\parsep}{\saveparskip}%
\setlength{\itemsep}{2pt} }}{\end{list}}

\newenvironment{twolist}{%
\begin{list}{{\rm (\arabic{ctr}.\arabic{ectr})}%
\hfill}{\usecounter{ectr} \labelwidth=26pt%
\labelsep=7pt \leftmargin=33pt \topsep=2pt%
\setlength{\listparindent}{\saveparindent}%
\setlength{\parsep}{\saveparskip}%
\setlength{\itemsep}{2pt} }}{\end{list}}

\newenvironment{centerlist}{%
\begin{list}{\mbox{}}{\labelwidth=0pt%
\labelsep=0pt \leftmargin=0pt \topsep=10pt%
\setlength{\listparindent}{\saveparindent}%
\setlength{\parsep}{\saveparskip}%
\setlength{\itemsep}{10pt} }}{\end{list}}

\newenvironment{newcenter}[1]{\begin{centerlist}\centering%
\item #1}{\end{centerlist}}

\newenvironment{codecenter}[1]{\begin{small}\begin{centerlist}\centering%
\item #1}{\end{centerlist}\end{small}}

% ========================================================================
% Math
% ========================================================================

\newlength{\savejot}
\setlength{\jot}{3pt}
\setlength{\savejot}{\jot}

\newenvironment{newmath}{\begin{displaymath}%
\setlength{\abovedisplayskip}{4pt}%
\setlength{\belowdisplayskip}{4pt}%
\setlength{\abovedisplayshortskip}{6pt}%
\setlength{\belowdisplayshortskip}{6pt} }{\end{displaymath}}

\newenvironment{newequation}{\begin{equation}%
\setlength{\abovedisplayskip}{4pt}%
\setlength{\belowdisplayskip}{4pt}%
\setlength{\abovedisplayshortskip}{6pt}%
\setlength{\belowdisplayshortskip}{6pt} }{\end{equation}}

\newenvironment{newalign}{\begin{align}%
\setlength{\abovedisplayskip}{4pt}%
\setlength{\belowdisplayskip}{4pt}%
\setlength{\abovedisplayshortskip}{6pt}%
\setlength{\belowdisplayshortskip}{6pt} }{\end{align}}

\newcommand{\bits}{\{0,1\}}
%\newcommand{\secparam}{\kappa}

\newcommand{\advA}{{\mathcal A}}
\newcommand{\advB}{{\mathcal B}}
\newcommand{\advC}{{\mathcal C}}
\newcommand{\advD}{{\mathcal D}}
\newcommand{\advE}{{\mathcal E}}

\newcommand{\calA}{{\cal A}}
\newcommand{\calB}{{\cal B}}
\newcommand{\calC}{{\cal C}}
\newcommand{\calD}{{\cal D}}
\newcommand{\calE}{{\cal E}}
\newcommand{\calF}{{\cal F}}
\newcommand{\calG}{{\cal G}}
\newcommand{\calH}{{\cal H}}
\newcommand{\calI}{{\cal I}}
\newcommand{\calJ}{{\cal J}}
\newcommand{\calK}{{\cal K}}
\newcommand{\calL}{{\cal L}}
\newcommand{\calM}{{\cal M}}
\newcommand{\calN}{{\cal N}}
\newcommand{\calO}{{\cal O}}
\newcommand{\calP}{{\cal P}}
\newcommand{\calQ}{{\cal Q}}
\newcommand{\calR}{{\cal R}}
\newcommand{\calS}{{\cal S}}
\newcommand{\calT}{{\cal T}}
\newcommand{\calU}{{\cal U}}
\newcommand{\calV}{{\cal V}}
\newcommand{\calW}{{\cal W}}
\newcommand{\calX}{{\cal X}}
\newcommand{\calY}{{\cal Y}}

\newcommand{\frakz}{{\mathfrak{z}}}

\newcommand{\N}{{{\mathbb N}}}
\newcommand{\Z}{{{\mathbb Z}}}
\newcommand{\bP}{{{\mathbb P}}}
\newcommand{\bE}{{{\mathbb E}}}
\newcommand{\B}{{{\textnormal B}}}
%\newcommand{\G}{{{\textnormal G}}}
\newcommand{\I}{{{\textnormal I}}}
\newcommand{\R}{{{\textnormal R}}}
\newcommand{\Gd}{{{\tilde{\mathbb G}}}}
\newcommand{\Y}{{{\sf Y}}}
%\newcommand{\R}{{{\rm\bf R}}}
\newcommand{\bigO}{\calO}
\newcommand{\goesto}{{\rightarrow}}
\newcommand{\eqdef}{\stackrel{\rm def}{=}}
\def\union{\cup}
\def\sep{\,}
\def\bigunion{\bigcup}
\def\suchthatt{\: :\:}
%\def\next{\hspace{12pt};\hspace{12pt}}
\def\nextt{\hspace{3pt};\hspace{6pt}}
\newcommand{\set}[2]{\{\:#1 \suchthatt #2\:\}}
\newcommand{\card}[1]{|#1|}
\def\leqq{\;\leq\;}
\def\eqq{\;=\;}
\def\geqq{\;\geq\;}
\def\lst{\;<\;}
\def\gst{\;>\;}
 \newcommand{\dash}{\mbox{-}}

\newcommand{\verylongleftarrow}[1]
      {\setlength{\unitlength}{.01in}
      \begin{picture}(#1,1) \put(#1,0){\vector(-1,0){#1}} \end{picture}}
\newcommand{\verylongrightarrow}[1]             %longleft and rightgoing arrows
      {\setlength{\unitlength}{.01in}           %for protocols
      \begin{picture}(#1,1) \put(0,0){\vector(1,0){#1}} \end{picture}}
\newcommand{\verylongbotharrow}[2]             %longleft and rightgoing arrows
      {\setlength{\unitlength}{.01in}           %for protocols
      \begin{picture}(#2,1) \put(#1,0){\vector(1,0){#1}}
                            \put(#1,0){\vector(-1,0){#1}} \end{picture}}
\newcommand{\leftgoing}[2]{{\stackrel{{\displaystyle #2}} {\verylongleftarrow{#1}}}}
\newcommand{\rightgoing}[2]{{\stackrel{{\displaystyle #2}} {\verylongrightarrow{#1}}}}
\newcommand{\bothgoing}[3]{{\stackrel{{\displaystyle #3}} {\verylongbotharrow{#1}{#2}}}}
\newcommand{\leftgoinga}[1]{\leftgoing{230}{#1} }
\newcommand{\rightgoinga}[1]{\rightgoing{230}{#1} }
\newcommand{\leftgoingb}[1]{\leftgoing{300}{#1} }
\newcommand{\rightgoingb}[1]{\rightgoing{300}{#1} }

\newcommand{\veryshortleftarrow}[1]
      {\setlength{\unitlength}{.01in}
      \begin{picture}(#1,1) \put(#1,0){\vector(-1,0){#1}} \end{picture}}

\newcommand{\getparse}[1]{{\:\stackrel{\raisebox{-0.5em}{{\hspace{0.1em}\mbox{\boldmath$\scriptscriptstyle #1$}}}}{\leftarrow}\:}}
\newcommand{\getu}{{\:\stackrel{\raisebox{-0.5em}{{\hspace{0.1em}\mbox{\boldmath$\scriptscriptstyle \cup$}}}}{\leftarrow}\:}}
\newcommand{\getdiff}{{\:\stackrel{{\scriptscriptstyle\hspace{0.2em} /}}{\leftarrow}\:}}
\newcommand{\getsr}{{\:{\leftarrow{\hspace*{-3pt}\raisebox{.75pt}{$\scriptscriptstyle\$$}}}\:}}
%\newcommand{\getsr}{{\:{\raisebox{3pt}{\veryshortleftarrow{15}}{\raisebox{1pt}{$\scriptscriptstyle\$$}}}\:}}
%\newcommand{\getsr}{{\:{\xleftarrow{\scriptscriptstyle\$}}\:}}
%\newcommand{\getsr}{{\:\stackrel{\raisebox{-0.5em}{$\scriptscriptstyle \hspace{0.2em}\$$}}{\leftarrow}\:}}
\newcommand{\sendsr}{{\:\stackrel{\scriptscriptstyle \hspace{0.2em}\$}{\rightarrow}\:}}
\renewcommand{\choose}[2]{{{#1}\atopwithdelims(){#2}}}
\newcommand{\?}{\stackrel{?}{=}}

%\DeclareMathOperator*{\argmax}{argmax}
%\DeclareMathOperator*{\argmin}{argmin}
%\DeclareMathOperator*{\prob}{\bP}
\DeclareMathOperator*{\ex}{\bE}

%\DeclarePairedDelimiter{\ceil}{\lceil}{\rceil}%
%\DeclarePairedDelimiter{\floor}{\lfloor}{\rfloor}%
%\DeclarePairedDelimiter\abs{\lvert}{\rvert}%
%\DeclarePairedDelimiter\norm{\lVert}{\rVert}%
\DeclarePairedDelimiter\prns{(}{)}%
\DeclarePairedDelimiter\braces{\{}{\}}%
\DeclarePairedDelimiter\bracks{[}{]}%
\DeclarePairedDelimiterX\condprns[2]{(}{)}{\,#1 \;\delimsize\vert\; #2\,}
\DeclarePairedDelimiterX\condbrks[2]{[}{]}{\,#1 \;\delimsize\vert\; #2\,}
\DeclarePairedDelimiterX\condbraces[2]{\{}{\}}{\,#1 \;\delimsize\vert\; #2\,}

% ========================================================================
% Game Code
% ========================================================================

\newcommand{\cif}{\mathbf{if\;}}
\newcommand{\cthen}{\mathbf{\;then\;}}
\newcommand{\celse}{\mathbf{else\;}}
\newcommand{\creturn}{\mathbf{return\;}}
\newcommand{\ctrue}{\mathsf{true}}
\newcommand{\cfalse}{\mathsf{false}}
\newcommand{\cbad}{\mathsf{bad}}
\newcommand{\cflag}{\mathsf{flag}}
\newcommand{\sets}{\mathsf{sets\;}}

\newcommand{\ind}{\hspace*{1em}}
\newcommand{\indsm}{\hspace*{.75em}}
\newcommand{\indeqn}{\;\;\;\;\;\;\;}

\newcommand{\query}[1]{\procfont{query} {#1}:}
\newcommand{\queryl}[1]{\underline{\procfont{query} {#1}:}}
%\newcommand{\oracle}[1]{\underline{\procfont{oracle} {#1}:}}
%\newcommand{\oraclev}[1]{\underline{\procfont{oracle} {#1}:}\smallskip}
%\newcommand{\procedure}[1]{\underline{\procfont{procedure} {#1}:}}
%\newcommand{\procedurev}[1]{\underline{\procfont{procedure} {#1}:}\smallskip}
\newcommand{\procedurev}[1]{\underline{{#1}:}\smallskip}
\newcommand{\subroutine}[1]{\underline{\procfont{subroutine} {#1}:}}
\newcommand{\subroutinev}[1]{\underline{\procfont{subroutine} {#1}:}\smallskip}
\newcommand{\subroutinenl}[1]{{\procfont{subroutine} {#1}:}}
\newcommand{\subroutinenlv}[1]{{\procfont{subroutine} {#1}:}\smallskip}
%\newcommand{\adversary}[1]{\underline{\procfont{adversary} {#1}:}}
\newcommand{\adversaryv}[1]{\underline{\procfont{adversary} {#1}:}\smallskip}
\newcommand{\experiment}[1]{\underline{{#1}}}
\newcommand{\experimentv}[1]{\underline{{#1}}\smallskip}
%\newcommand{\algorithm}[1]{\underline{\procfont{algorithm} {#1}:}}
\newcommand{\algorithmv}[1]{\underline{\procfont{algorithm} {#1}:}\smallskip}
%\newcommand{\experiment}[1]{\underline{\procfont{Experiment} {#1}}}
%\newcommand{\experimentv}[1]{\underline{\procfont{Experiment} {#1}}\smallskip}d

\newcommand{\gamesfontsize}{\small}
\newcommand{\stretchval}{1.2}

\newcommand{\mpage}[2]{\begin{minipage}{#1\textwidth} #2 \end{minipage}}
\newcommand{\fpage}[2]{\framebox{\begin{minipage}{#1\textwidth}\setstretch{\stretchval}\gamesfontsize #2 \end{minipage}}}

\newcommand{\hfpages}[3]{\hfpagess{#1}{#1}{#2}{#3}}
\newcommand{\hfpagess}[4]{
		\begin{tabular}{c@{\hspace*{.5em}}c}
		\framebox{\begin{minipage}[t]{#1\textwidth}\setstretch{\stretchval}\gamesfontsize #3 \end{minipage}}
		&
		\framebox{\begin{minipage}[t]{#2\textwidth}\setstretch{\stretchval}\gamesfontsize #4 \end{minipage}}
		\end{tabular}
	}
\newcommand{\hfpagesss}[6]{
		\begin{tabular}{c@{\hspace*{.5em}}c@{\hspace*{.5em}}c}
		\framebox{\begin{minipage}[t]{#1\textwidth}\setstretch{\stretchval}\gamesfontsize #4 \end{minipage}}
		&
		\framebox{\begin{minipage}[t]{#2\textwidth}\setstretch{\stretchval}\gamesfontsize #5 \end{minipage}}
		&
		\framebox{\begin{minipage}[t]{#3\textwidth}\setstretch{\stretchval}\gamesfontsize #6 \end{minipage}}
		\end{tabular}
	}
\newcommand{\hfpagessss}[8]{
		\begin{tabular}{c@{\hspace*{.5em}}c@{\hspace*{.5em}}c@{\hspace*{.5em}}c}
		\framebox{\begin{minipage}[t]{#1\textwidth}\setstretch{\stretchval}\gamesfontsize #5 \end{minipage}}
		&
		\framebox{\begin{minipage}[t]{#2\textwidth}\setstretch{\stretchval}\gamesfontsize #6 \end{minipage}}
		&
		\framebox{\begin{minipage}[t]{#3\textwidth}\setstretch{\stretchval}\gamesfontsize #7 \end{minipage}}
		&
		\framebox{\begin{minipage}[t]{#4\textwidth}\setstretch{\stretchval}\gamesfontsize #8 \end{minipage}}
		\end{tabular}
	}

\newcommand{\hfpagesssss}[6]{
		\begin{tabular}{c@{\hspace*{.5em}}c@{\hspace*{.5em}}c@{\hspace*{.5em}}c@{\hspace*{.5em}}c}
		\framebox{\begin{minipage}[t]{#1\textwidth}\setstretch{\stretchval}\gamesfontsize #2 \end{minipage}}
		&
		\framebox{\begin{minipage}[t]{#1\textwidth}\setstretch{\stretchval}\gamesfontsize #3 \end{minipage}}
		&
		\framebox{\begin{minipage}[t]{#1\textwidth}\setstretch{\stretchval}\gamesfontsize #4 \end{minipage}}
		&
		\framebox{\begin{minipage}[t]{#1\textwidth}\setstretch{\stretchval}\gamesfontsize #5 \end{minipage}}
		&
		\framebox{\begin{minipage}[t]{#1\textwidth}\setstretch{\stretchval}\gamesfontsize #6 \end{minipage}}
		\end{tabular}
	}

\newcommand{\fhpagess}[4]{
	\framebox{
		\begin{tabular}{c@{\hspace*{.5em}}c@{\hspace*{.5em}}c}
			\begin{minipage}[t]{#1\textwidth}\setstretch{\stretchval}\gamesfontsize #3 \end{minipage}
			&
			\begin{minipage}[t]{#2\textwidth}\setstretch{\stretchval}\gamesfontsize #4 \end{minipage}
		\end{tabular}
	}
}

\newcommand{\fhpagesss}[6]{
	\framebox{
    \begin{tabular}[t]{c@{\hspace*{.5em}}c@{\hspace*{.5em}}c}
     % \adjustbox{valign=c}{
      \begin{minipage}[t]{#1\textwidth}\setstretch{\stretchval}\gamesfontsize #4 \end{minipage}
    %  }
			&
    %  \adjustbox{valign=c}{
      \begin{minipage}[t]{#2\textwidth}\setstretch{\stretchval}\gamesfontsize #5 \end{minipage}
     % }
			&
     % \adjustbox{valign=c}{
      \begin{minipage}[t]{#3\textwidth}\setstretch{\stretchval}\gamesfontsize #6 \end{minipage}
     % }
		\end{tabular}
	}
}

\newcommand{\fhpagessss}[7]{
    \framebox{
		\begin{tabular}{c@{\hspace*{.5em}}c@{\hspace*{.5em}}c}
	  \multicolumn{3}{l}{\gamesfontsize #7}\\
		\begin{minipage}[t]{#1\textwidth}\setstretch{\stretchval}\gamesfontsize #4 \end{minipage}
		&
		\begin{minipage}[t]{#2\textwidth}\setstretch{\stretchval}\gamesfontsize #5 \end{minipage}
    &
		\begin{minipage}[t]{#3\textwidth}\setstretch{\stretchval}\gamesfontsize #6 \end{minipage}
		\end{tabular}
    }
	}

\def\codestretch{\stretchval}

\newcommand{\hpagesl}[3]{
	\begin{tabular}{c|c}
	  \begin{minipage}{#1\textwidth}\setstretch{\codestretch} #2 \end{minipage}
	  &
	  \begin{minipage}{#1\textwidth} #3 \end{minipage}
	\end{tabular}
	}

\newcommand{\hpagessl}[4]{
	\begin{tabular}{c|@{\hspace*{.5em}}c}
	   \begin{minipage}[t]{#1\textwidth}\setstretch{\codestretch} #3 \end{minipage}
	   &
	   \begin{minipage}[t]{#2\textwidth}\setstretch{\codestretch} #4 \end{minipage}
	\end{tabular}
	}

\newcommand{\hpages}[3]{
	\begin{tabular}{cc}
	   \begin{minipage}[t]{#1\textwidth}\setstretch{\codestretch} #2 \end{minipage}
	   &
	   \begin{minipage}[t]{#1\textwidth}\setstretch{\codestretch} #3 \end{minipage}
	\end{tabular}
	}

\newcommand{\hpagess}[4]{
    \begin{tabular}[t]{c@{\hspace*{1.5em}}c}
	   \adjustbox{valign=c}{\begin{minipage}[t]{#1\textwidth}\setstretch{\codestretch} #3 \end{minipage}}
	   &
     \adjustbox{valign=c}{\begin{minipage}[t]{#2\textwidth}\setstretch{\codestretch} #4 \end{minipage}}
    \end{tabular}
	}

\newcommand{\hpagesss}[6]{
  \begin{tabular}[t]{c@{\hspace*{.5em}}c@{\hspace*{.5em}}c}
      \adjustbox{valign=c}{\begin{minipage}[t]{#1\textwidth}\setstretch{\stretchval}\gamesfontsize #4 \end{minipage}}
			&
      \adjustbox{valign=c}{\begin{minipage}[t]{#2\textwidth}\setstretch{\stretchval}\gamesfontsize #5 \end{minipage}}
			&
      \adjustbox{valign=c}{\begin{minipage}[t]{#3\textwidth}\setstretch{\stretchval}\gamesfontsize #6 \end{minipage}}
	\end{tabular}}

\newcommand{\hpagesssl}[6]{
	\begin{tabular}{c|c|c}
	\begin{minipage}[t]{#1\textwidth}\setstretch{\codestretch} #4 \end{minipage} &
	\begin{minipage}[t]{#2\textwidth}\setstretch{\codestretch} #5 \end{minipage} &
	\begin{minipage}[t]{#3\textwidth}\setstretch{\codestretch} #6 \end{minipage}
	\end{tabular}}
\newcommand{\hpagessss}[8]{
	\begin{tabular}{cccc}
	\begin{minipage}[t]{#1\textwidth}\setstretch{\codestretch} #5 \end{minipage} &
	\begin{minipage}[t]{#2\textwidth}\setstretch{\codestretch} #6 \end{minipage} &
	\begin{minipage}[t]{#3\textwidth}\setstretch{\codestretch} #7 \end{minipage}
	\begin{minipage}[t]{#4\textwidth}\setstretch{\codestretch} #8 \end{minipage}
	\end{tabular}}

\renewcommand{\paragraph}[1]{\vspace{.6em}\noindent\textbf{#1.}\hspace*{.5em}}
% ========================================================================
% Author Notes
% ========================================================================
\iffalse
%%PAG: I'm commenting these out so they don't conflict with others in zz_header
\def\authors{1}
\def\authnotes{0}
\newcommand{\authnote}[2]{\ifnum\authnotes=1\begin{quote}\textbf{#1 says:} #2\end{quote}\fi}
\newcommand{\authfnote}[2]{\ifnum\authnotes=1\footnote{\textbf{#1 says:} #2}\fi}
\newcommand{\myemph}[1]{\textsl{\textbf{#1}}}


\newcounter{mynote}[section]
\definecolor{darkcyan}{rgb}{0.0, 0.55, 0.55}
\definecolor{darkmagenta}{rgb}{0.55, 0.0, 0.55}
\definecolor{darkpink}{rgb}{0.86,0.44,0.58}
\definecolor{green(pigment)}{rgb}{0.0, 0.65, 0.31}
\newcommand{\notecolor}{blue}
\newcommand{\quangnotecolor}{green(pigment)}

\newcommand{\paulnotecolor}{darkcyan}

\newcommand{\jimnotecolor}{magenta}

\newcommand{\opalnotecolor}{orange}

\newcommand{\thenote}{\thesection.\arabic{mynote}}
\newcommand{\qnote}[1]{\ifnum\authnotes=1\refstepcounter{mynote}{\bf \textcolor{\quangnotecolor}{$\ll$QD~\thenote: {\sf #1}$\gg$}}\fi}

\newcommand{\paulnote}[1]{\ifnum\authnotes=1\refstepcounter{mynote}{\bf \textcolor{\paulnotecolor}{$\ll$PG~\thenote: {\sf #1}$\gg$}}\fi}

\newcommand{\jimnote}[1]{\ifnum\authnotes=1\refstepcounter{mynote}{\bf \textcolor{\jimnotecolor}{$\ll$JM~\thenote: {\sf #1}$\gg$}}\fi}
\newcommand{\opalnote}[1]{\ifnum\authnotes=1\refstepcounter{mynote}{\bf \textcolor{\opalnotecolor}{$\ll$OW~\thenote: {\sf #1}$\gg$}}\fi}


\newcommand{\todo}[1]{\ifnum\authnotes=1\refstepcounter{mynote}{\bf \textcolor{red}{$\ll$TODO: \thenote: {\sf #1}$\gg$}}\fi}
\newcommand{\fixme}[1]{\ifnum\authnotes=1\textbf{\textcolor{red}{[FIXME: #1]}}\fi}
\newcommand{\checkme}[1]{\ifnum\authnotes=1\textbf{\textcolor{red}{[CHECKME: #1]}}\fi}
% \newcommand{\xxx}{\ifnum\authnotes=1\textbf{\textcolor{red}{[XXX]}}\fi}
\newcommand{\xxx}{\textbf{\textcolor{red}{[XXX]}}\typeout{Missing number}}
\newcommand{\rrr}[1]{\textbf{\textcolor{red}{[ref #1]}}\typeout{Missing number}}
\newcommand{\ccc}[1]{\textbf{\textcolor{red}{[cite #1]}}\typeout{Missing number}}
\newcommand{\addcite}{\ifnum\authnotes=1\textbf{\textcolor{red}{[CITE]}}\fi}
\newcommand{\textred}[1]{\textcolor{red}{#1}}
\newcommand{\ignore}[1]{}

% definitions for marginpars
\def\hn{\sffamily\selectfont}
\newcommand{\mpfont}{\hn\scriptsize}
\setlength\marginparwidth{36pt}
\setlength\marginparsep{5pt}

\newcommand{\MPworker}[2]{\ifnum\authnotes=1\unskip{\color{#1}\vrule\vrule}{\marginpar{\raggedright\color{#1}\mpfont #2}}\fi}

\newcommand{\QP}[1]{\MPworker{\mikenotecolor}{QD: #1}}
\newcommand{\PGP}[1]{\MPworker{\paulnotecolor}{PAG: #1}}



\usepackage{ulem}
\normalem

% changebars
\ifnum\authnotes=1
    \newcommand{\changebars}[2]{%
    [{\color{magenta}\em\begingroup{#1}\endgroup}][{\color{magenta}\sout{#2}}]}
\else
    \newcommand{\changebars}[2]{#1}
\fi

\newcommand{\changebarsii}[2]{#1}

\fi
\renewcommand{\ttdefault}{pxtt}   %% better \tt font
% ========================================================================
% Basic Crypto Notation
% ========================================================================

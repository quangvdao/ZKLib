\section{Introduction}\label{sec:intro}

\paragraph{Why formal verification of SNARK implementations?}

\paragraph{Why zkVMs (and Jolt in general)?}

De-duplicating efforts to prove soundness of individual circuits, instead having a ``universal'' approach that allows to prove any program that may be compiled down to virtual machine instructions.

Among the space of zkVMs, Jolt has one of, if not, \emph{the} best-performing prover. It is also based on techniques that have not seen formal verification before, such as multilinear polynomials, sumcheck-based protocols, and binary tower fields (for integration with the Binius polynomial commitment scheme).

\paragraph{Challenges in formally verifying Jolt}

\begin{itemize}
    \item Moving target, since Jolt is still under development
    \item A suite of protocols that have not been formally verified before
\end{itemize}

\subsection{Design Decisions}

% just a running draft of things to say in the final paper

\paragraph{Why Lean?}

\begin{itemize}
    \item Both a programming language and a theorem prover. Other tools, such as Easycrypt or Coq, requires code extraction for executable code.
    \item Contains Mathlib, a huge undertaking by mathematicians to formalize modern mathematics. Of interests to us are results about finite fields and multivariate polynomials.
\end{itemize}

The same justifications were given in prior papers (Bailey-Miller and Bailey-Mishra).

\paragraph{Structure of the Formalization}

We define the following 


\subsection{Related Works}\label{sec:related-works}

Prior SNARK verifications:
\begin{itemize}
    \item Bailey-Miller formalizes the soundness of Linear-PCP based SNARKs. Their proofs do not come with executable code.
    \item Bailey-Mishra formalizes the soundness (with executable code) for Marlin, a SNARK based on univariate polynomials. In contrast, our formalization deals with multilinear- and multivariate-based protocols.
\end{itemize}

Formal verification of sumcheck in Isabelle:

Verification of other aspects of SNARKs:
\begin{itemize}
    \item SNARK circuits and relations [citations] 
\end{itemize}
These are complementary to our work.

Other projects on cryptography meets formal verification: [give citations & quick descriptions]